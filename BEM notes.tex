%% LyX 2.3.6.1 created this file.  For more info, see http://www.lyx.org/.
%% Do not edit unless you really know what you are doing.
\documentclass[12pt,a4paper]{article}
\usepackage[T1]{fontenc}
\usepackage{CJKutf8}
\synctex=-1
\usepackage{color}
\usepackage{amsmath}
\usepackage{stmaryrd}
\usepackage{setspace}
\usepackage{wasysym}
\doublespacing
\usepackage[unicode=true,
 bookmarks=true,bookmarksnumbered=true,bookmarksopen=true,bookmarksopenlevel=2,
 breaklinks=false,pdfborder={0 0 1},backref=false,colorlinks=true]
 {hyperref}
\hypersetup{pdftitle={Introduction to LyX},
 pdfauthor={LyX Team},
 pdfsubject={LyX-documentation Intro},
 pdfkeywords={LyX, documentation},
 linkcolor=black, citecolor=black, urlcolor=blue, filecolor=blue,pdfpagelayout=OneColumn, pdfnewwindow=true, pdfstartview=XYZ, plainpages=false}

\makeatletter

%%%%%%%%%%%%%%%%%%%%%%%%%%%%%% LyX specific LaTeX commands.
\pdfpageheight\paperheight
\pdfpagewidth\paperwidth


%%%%%%%%%%%%%%%%%%%%%%%%%%%%%% User specified LaTeX commands.
\usepackage{ifxetex}
\ifxetex
\usepackage{xeCJK}
\fi

\usepackage{indentfirst}
\renewcommand\abstractname{摘要}
\renewcommand\refname{参考文献}

\makeatother

\begin{document}
\begin{CJK}{UTF8}{gbsn}%
\title{\textbf{Boundary Element Method}}
\author{Gong Zekang}
\maketitle

\section{Governing Equations}

Equation of motion for elastic waves:
\begin{equation}
(\lambda+\mu)u_{j,ji}+\mu u_{i,jj}+f_{i}=\rho\ddot{u_{i}}
\end{equation}

where $\lambda$ and $\mu$ are Lame constants, $u_{i}$ is the displacement
vector, $f_{i}$ is the body force applied to the mass point and $\rho$
is the density of the medium.

\section{Boundary Integral}

\subsection{Elastic Medium}

\subsubsection*{Betti Reciprocal Theorem}

Betti reciprocal theorem (without considering the inertial force):
In the elastic domain $D$ with boundary $S$, let $u_{i}$ and $u_{i}^{*}$
be the displacement fields from the body force $f_{i}$ and $f_{i}^{*}$,
respectively. The two fields exhibit the relation:
\[
\int_{D}f_{i}^{*}u_{i}dD+\int_{S}t_{i}^{*}u_{i}dS=\int_{D}f_{i}u_{i}^{*}dD+\int_{S}t_{i}u_{i}^{*}dS
\]

yields

\begin{equation}
\int_{D}f_{i}^{*}u_{i}dD-\int_{D}f_{i}u_{i}^{*}dD+=\int_{S}t_{i}u_{i}^{*}dS-\int_{S}t_{i}^{*}u_{i}dS.
\end{equation}


\subsubsection*{Introduction of Green's Function}

Now we suppose that displacement field $u_{i}^{*}$ is exicited by
a unit pulse force in the $j$ direction at point $x^{'}$ in domain
$D$. The force can be expressed as fucntion:
\[
f_{i}^{*}=\delta_{ij}\delta(x-x')
\]
where $\delta_{ij}$ is the Kronecker delta ande $\delta(x-x')$ is
the Dirac function.

The displacement field excited by this single source is called the
Green's function in elastodynamics and is written as $G_{ij}(x;x')$.
From equation (1), there is 

\[
(\lambda+\mu)G_{kj,ik}+\mu G_{ij,kk}-\rho\ddot{G_{ij}}=-\delta_{ij}\delta(x-x').
\]


\subsubsection*{Direct BEM and Singularity on Boundary S}

According to the property of the Dirac function, we have
\begin{equation}
\int_{D}u_{i}\delta_{ij}\delta(x-x')dD_{x'}=\delta_{ij}u_{i}(x).
\end{equation}

From equation (2), replacing $u^{*}$, $t^{*}$ by $G_{ij}$, $T_{ij}$
respectively , we deduce
\[
\int_{D}u_{i}\delta_{ij}\delta(x-x')dD_{x'}-\int_{D}G_{ij}f_{i}dD=\int_{S}(G_{ij}t_{i}-u_{i}T_{ij})dS,
\]

and using equation (3), we arrive at
\begin{equation}
\delta_{ij}u_{i}(x)=\int_{S}[G_{ij}(x;x')t_{i}(x)-u_{i}T_{ij}(x,x')]dS_{x'}+\int_{D}G_{ij}(x;x')f_{i}dD
\end{equation}

which is derived under the condition of $x'$ being inside $D$.

Expressions such as eq.(4) for the displacement within region $D$,
that involve integrals like
\begin{equation}
\int_{S}G_{ij}\psi_{i}ds_{x}
\end{equation}
and
\begin{equation}
\int_{S}T_{ij}\psi_{i}ds_{x}
\end{equation}

The Green's function $G_{ij}$ in 2-D static P-SV motion is given
by
\[
G_{ij}^{s}(x;x')=\dfrac{-1}{8\pi\mu(1-v)}[(3-4v)\delta_{ij}\log r-\hat{r_{i}}\hat{r_{j}}]+A_{ij},
\]

where $v$ is Poisson's ratio, $r=|x-x'|,$ $\hat{r_{i}}=(x_{i}-x_{i}')/r$,
and $A_{ij}$ is a constant tensor. There is therefore a logarithmic
singularity in $G_{ij}$ at $x=x'$, but this is integrable.

Considering eq.(6), $T_{ij}$ is given by

\[
T_{ij}(x;x')=c_{ipkl}\hat{n_{p}}(x)\dfrac{\partial}{\partial x_{l}}G_{kj}(x;x'),
\]

where $\hat{n}$ is the outward normal to the boundary $S$. The static
equivalent of this is
\[
T_{ij}^{s}(x;x')=\dfrac{-1}{4\pi r(1-v)}\{(1-2v)(\hat{n}_{j}\hat{r}_{i}-\hat{n}_{i}\hat{r}_{j})+[(1-2v)\delta_{ij}+2\hat{r}_{i}\hat{r}_{j}]\hat{r}_{k}\hat{n}_{k}\},
\]

and this has a $(\dfrac{1}{r})$ singularity, which is not integrable
when $x'=x^{S}\in S.$

Consider the point $x'$ within the region $D$ as it approaches a
point $x^{S}$on the boundary $S.$We split the curve. There we have,
\[
S=S'+S_{\epsilon},
\]

where $S_{\epsilon}$ is a section of curve centred on $x^{S}$ and
length of $\epsilon$ on either side. Thus the main task is to evaluate
\begin{equation}
\underset{x'\shortrightarrow x^{S}}{\lim}\int_{S_{\epsilon}}T_{ij}\psi_{i}ds_{x}=\underset{x'\shortrightarrow x^{S}}{\lim}\{\psi_{i}(x^{S})\int_{S_{\epsilon}}T_{ij}(x;x')ds_{x}+\int_{S_{\epsilon}}T_{ij}(x;x')[\psi_{i}(x)-\psi(x^{S})]ds_{x}\}.
\end{equation}

If $\psi$ is Hölder continuous on S, there we have
\[
|\psi_{i}(x_{1})-\psi_{i}(x_{2})|\le L|x_{1}-x_{2}|^{\alpha},
\]

where $0<\alpha\le1$. On the assumption that this inequality holds,
the second integral in eq. (7) is integrable and bounded when $x'=x^{S}$.
Thus the main task is to evaluate the integral $\int_{S_{\epsilon}}T_{ij}(x;x')ds_{x}$.

We have assumed that $S$ is smooth and so we may replace the curve
$S_{\epsilon}$ by a section of straight line, tangent to $S_{\epsilon}$
at $x^{S}$ with error of order of $\epsilon$: 
\[
\int_{S_{\epsilon}}T_{ij}(x;x')ds_{x}=\int_{-\epsilon}^{\epsilon}T_{ij}(x;x')+O(\epsilon),
\]
$x'=(0,\eta),\hat{n}(x)=(0,-1).$ 

If let $x'\shortrightarrow x^{S}$,the integral $\int_{S_{\epsilon}}T_{ij}(x;x')$
of static form for $T_{ij}$ is given by
\[
\int_{S_{\epsilon}}T_{ij}(x;x')ds_{x}=-\dfrac{1}{2}\delta_{ij}+O(\epsilon).
\]


\subsubsection*{Indirect BEM}

\subsection{Fluid Medium}

Consider the linear acoustic first-order wave equation:

\[
\dfrac{1}{K}\dot{p}+\nabla\cdot v=f,
\]

and Newton's Second Law:

\[
\rho\dot{v}+\nabla p=0,
\]

where $v$ is velocity field and $f$ is the loading.

Consider the dispalcement potential function $\psi$ that satisfy
the following diferential relations:
\[
u=\nabla\psi
\]
\[
p=-\rho\ddot{\psi}.
\]

The scalar wave equation can be expressed as:
\begin{equation}
[\dfrac{1}{c^{2}}\ddot{\psi}-\nabla^{2}\psi]_{t}=f,
\end{equation}

where $c=\dfrac{K}{\rho}$ is the wave velocity. Integrating eq.(8),
we have second order scalar wave equation:
\[
\dfrac{1}{c^{2}}\ddot{\psi}-\nabla^{2}\psi=F.
\]

Define Fourier transform as
\[
\hat{u}(\omega)=\int_{-\infty}^{+\infty}u(t)e^{i\omega t}dt,
\]

and Fourier transform the wave equation, we have
\begin{equation}
-\dfrac{\omega^{2}}{c^{2}}\hat{\psi}-\nabla^{2}\hat{\psi}=\hat{F.}
\end{equation}

The Green's function for this problem satisfies 
\[
-\dfrac{\omega^{2}}{c^{2}}\hat{G}(x;x')-\nabla^{2}\hat{G}(x;x')=\delta(x-x').
\]

Multiply the Green's function onto the eq.(9), and integrate over
the finite volume $V$, we have
\[
\int_{V}-\dfrac{\omega^{2}}{c^{2}}\psi G-G\nabla^{2}\psi-GFdV_{x}=0.
\]

Then, apply the Green's second identity
\[
\int_{V}G\nabla^{2}\psi-\psi\nabla^{2}GdV=\int_{\partial V}\boldsymbol{n}\cdot(G\nabla\psi-\psi\nabla G)dS,
\]

and substitute the term $G\nabla^{2}\psi$, we have

\[
\int_{V}-\dfrac{\omega^{2}}{c^{2}}\psi G-GFdV_{x}=\int_{V}\psi\nabla^{2}GdV+\int_{\partial V}\boldsymbol{n}\cdot(G\nabla\psi-\psi\nabla G)dS_{x},
\]

then rearrange this equation, we have
\[
\int_{V}\psi[-\dfrac{\omega^{2}}{c^{2}}G-\nabla^{2}G]dV_{x}=\int_{V}GFdV+\int_{\partial V}\boldsymbol{n}\cdot(G\nabla\psi-\psi\nabla G)dS_{x},
\]

Use the property of the Green's function:
\[
\psi(x)=\int_{V}\psi(x)\delta(x-x')dV_{x}=\int_{V}GFdV_{x}+\int_{\partial V}\boldsymbol{n}\cdot(G\nabla\psi-\psi\nabla G)dS_{x}
\]

The term in absence of sources,
\[
\psi(x)=\int_{\partial V}\boldsymbol{n}\cdot(G\nabla\psi-\psi\nabla G)dS_{x}
\]


\section{Green's Function}

\subsection{Contitutive Model}

\[
T_{ij}(x,x')=c_{ipkl}\hat{n}_{p}(x)\dfrac{\partial}{\partial x_{l}}G_{kj}(x,x'),
\]

where $\hat{n}(x)$ is the outward normal vector to the boundary $S.$
And $c_{ipkl}=\lambda\delta_{ip}\delta_{kl}+\mu(\delta_{ik}\delta_{pl}+\delta_{il}\delta_{pk}),$where
$\lambda=\dfrac{2v\mu}{1-2v}$. 

\subsection{Green's Function of Displacement and Traction}

For 2D in plane (P-SV) problem, 
\[
G_{ij}=
\]

$\dfrac{\partial H_{0}^{(2)}(kr)}{\partial x_{j}}=-k\hat{r}_{j}H_{1}^{(2)}(kr)$,

$\dfrac{\partial r}{\partial x_{j}}=\hat{r}_{j},$

$\dfrac{\partial^{2}r}{\partial x_{j}\partial x_{l}}=\hat{r}_{jl}=-\dfrac{\hat{r}_{j}\halfnote\hat{r}_{l}}{r}+\delta_{jl}\dfrac{1}{r},$

in order to avoid singularity, we use $\dfrac{\partial H_{1}^{(2)}(x)}{\partial x}=\dfrac{1}{2}(H_{0}^{(2)}(x)-H_{2}^{(2)}(x))$.

$\dfrac{\partial H_{1}^{(2)}(kr)}{\partial x_{j}}=k\hat{r}_{j}\dfrac{1}{2}(H_{0}^{(2)}(kr)-H_{2}^{(2)}(kr)),$

\begin{eqnarray*}
T_{ij}(x,x') & = & c_{ipkl}\hat{n}_{p}(x)\dfrac{\partial}{\partial x_{l}}G_{kj}(x,x')
\end{eqnarray*}

\[
=c_{ipkl}\hat{n}_{p}(x)\dfrac{i}{4\mu}\{-\delta_{kj}k_{\beta}\hat{r}_{l}H_{1}^{(2)}(k_{\beta}r)-\dfrac{1}{k_{\beta}}(\dfrac{\partial}{\partial x_{l}}\dfrac{1}{r}\hat{r}_{k}\hat{r}_{j}+\dfrac{1}{r}\hat{r}_{k}\hat{r}_{jl}+\dfrac{1}{r}\hat{r}_{j}\hat{r}_{kl})[H_{1}^{(2)}(k_{\beta}r)-\dfrac{\alpha}{\beta}H_{1}^{(2)}(k_{\alpha}r)]
\]

\[
-\dfrac{1}{2k_{\beta}r}\hat{r}_{k}\hat{r}_{j}\hat{r}_{l}[k_{\beta}H_{0}^{(2)}(k_{\beta}r)-k_{\beta}H_{2}^{(2)}(k_{\beta}r)-\dfrac{\beta}{\alpha}k_{\alpha}H_{2}^{(2)}(k_{\alpha}r)+\dfrac{\beta}{\alpha}k_{\alpha}H_{0}^{(2)}(k_{\alpha}r)]\}
\]

\[
-\dfrac{i}{4\mu}\{(\hat{r}_{k}\hat{r}_{jl}+\hat{r}_{j}\hat{r}_{kl})[H_{0}^{(2)}(k_{\beta}r)-\dfrac{\beta^{2}}{\alpha^{2}}H_{0}^{(2)}(k_{\alpha}r)]+\hat{r}_{k}\hat{r}_{j}\hat{r}_{l}[\dfrac{\beta^{2}}{\alpha^{2}}k_{\alpha}H_{1}^{(2)}(k_{\alpha}r)-k_{\beta}H_{1}^{(2)}(k_{\beta}r)]\}.
\]
Considering $T_{nk},$ where suffix $n$ stands for normal vector
outwards the boundary, $T_{nk}=$$\hat{n}_{l}T_{lk}$. ( $\hat{n}=(cos\theta,sin\theta)$
). 

\begin{eqnarray*}
T_{ij}(x,x') & = & c_{ipkl}\hat{n}_{p}(x)\dfrac{\partial}{\partial x_{l}}G_{kj}(x,x')
\end{eqnarray*}

\[
=c_{ipkl}\hat{n}_{p}(x)\dfrac{i}{4\mu}\{-\delta_{kj}k_{\beta}\hat{r}_{l}H_{1}^{(2)}(k_{\beta}r)-\dfrac{1}{k_{\beta}}(\dfrac{\partial}{\partial x_{l}}\dfrac{1}{r}\hat{r}_{k}\hat{r}_{j}+\dfrac{1}{r}\hat{r}_{k}\hat{r}_{jl}+\dfrac{1}{r}\hat{r}_{j}\hat{r}_{kl})[H_{1}^{(2)}(k_{\beta}r)-\dfrac{\alpha}{\beta}H_{1}^{(2)}(k_{\alpha}r)]
\]

\[
-\dfrac{1}{2k_{\beta}r}\hat{r}_{k}\hat{r}_{j}\hat{r}_{l}[k_{\beta}H_{0}^{(2)}(k_{\beta}r)-k_{\beta}H_{2}^{(2)}(k_{\beta}r)-\dfrac{\beta}{\alpha}k_{\alpha}H_{2}^{(2)}(k_{\alpha}r)+\dfrac{\beta}{\alpha}k_{\alpha}H_{0}^{(2)}(k_{\alpha}r)]\}
\]

\[
-\dfrac{i}{4\mu}\{(\hat{r}_{k}\hat{r}_{jl}+\hat{r}_{j}\hat{r}_{kl})[H_{0}^{(2)}(k_{\beta}r)-\dfrac{\beta^{2}}{\alpha^{2}}H_{0}^{(2)}(k_{\alpha}r)]+\hat{r}_{k}\hat{r}_{j}\hat{r}_{l}[\dfrac{\beta^{2}}{\alpha^{2}}k_{\alpha}H_{1}^{(2)}(k_{\alpha}r)-k_{\beta}H_{1}^{(2)}(k_{\beta}r)]\}.
\]
Considering $T_{nk},$ where suffix $n$ stands for normal vector
outwards the boundary, $T_{nk}=$$\hat{n}_{l}T_{lk}$. ( $\hat{n}=(cos\theta,sin\theta)$
). 

\section{viscous fluid}

\subsection{Governing equation}

Governing Equation:

\[
\boldsymbol{u}_{,tt}-c_{p}^{2}\nabla(\nabla\cdot\boldsymbol{u})-\dfrac{\mu}{\rho}\nabla^{2}\boldsymbol{u}_{,t}=f
\]

\[
\omega^{2}\boldsymbol{u}+(c_{p}^{2}-\dfrac{i\omega\mu}{\rho})\nabla(\nabla\cdot\boldsymbol{u})+\dfrac{i\omega\mu}{\rho}\nabla\times(\nabla\times\boldsymbol{u})=\boldsymbol{-f}
\]

\[
\omega^{2}\boldsymbol{u}+\hat{c}_{p}^{2}\nabla^{2}\boldsymbol{u}-\hat{c}_{s}^{2}\nabla\times(\nabla\times\boldsymbol{u})=\boldsymbol{-f}
\]

where $c_{p}=\sqrt{\dfrac{K}{\rho}}$, $\hat{c}_{p}=\sqrt{\dfrac{-i\omega\mu}{\rho}+c_{p}^{2}}$,
$\hat{c}_{s}=\sqrt{\dfrac{-i\omega\mu}{\rho}}$.

Green's function for 2-D LNS:

\begin{align*}
G_{ij} & =\dfrac{i}{4\mu}\{\delta_{ij}H_{0}^{(2)}(k_{s}r)-\dfrac{1}{k_{s}}\dfrac{\partial r}{\partial x_{i}}\dfrac{\partial r}{\partial x_{j}}[H_{1}^{(2)}(k_{s}r)-\dfrac{\hat{c}_{s}}{\hat{c}_{p}}H_{1}^{(2)}(k_{p}r)]\}\\
 & -\dfrac{i}{4\mu}\{\text{\ensuremath{\dfrac{\partial r}{\partial x_{i}}\dfrac{\partial r}{\partial x_{j}}}}[H_{0}^{(2)}(k_{s}r)-\dfrac{\hat{c}_{s}^{2}}{\hat{c}_{p}^{2}}H_{0}^{(2)}(k_{s}r)]\}
\end{align*}

On the boundary $S$, 
\[
u_{total}=u_{incident}+u_{diffracted}=u_{refracted}
\]

Thus we have
\[
u_{j}^{E}(\boldsymbol{x})-u_{j}^{I}(\boldsymbol{x})=-u_{j}^{(i)}(\boldsymbol{x})
\]

\[
t_{j}^{E}(\boldsymbol{x})-t_{j}^{I}(\boldsymbol{x})=-t_{j}^{(i)}(\boldsymbol{x}),\ \boldsymbol{x}\in S
\]

where $E$ and $I$ indicates the (displacment \& traction) response
casused by boundary $S$ alone.

The discrete version is:
\[
\sum_{l=1}^{M}\bar{G}_{jk}^{E}(\boldsymbol{x}_{m},\boldsymbol{\xi}_{l})\phi_{kl}^{E}-\sum_{l=1}^{M}\bar{G}_{jk}^{I}(\boldsymbol{x}_{m},\boldsymbol{\xi}_{l})\phi_{kl}^{I}=-u_{j}^{(i)}(\boldsymbol{x}_{m}),\ m=1...M
\]

\[
\sum_{l=1}^{M}\bar{T}_{jk}^{E}(\boldsymbol{x}_{m},\boldsymbol{\xi}_{l})\phi_{kl}^{E}-\sum_{l=1}^{M}\bar{T}_{jk}^{I}(\boldsymbol{x}_{m},\boldsymbol{\xi}_{l})\phi_{kl}^{I}=-t_{j}^{(i)}(\boldsymbol{x}_{m}),\ m=1...M
\]


\subsection{Boundary Integral Representation}

\[
\bar{G}_{jk}^{E}(\boldsymbol{x}_{m},\boldsymbol{\xi}_{l})=\int_{\Delta S_{l}}G_{jk}^{E}(\boldsymbol{x}_{m},\boldsymbol{\xi})dS_{\boldsymbol{\xi}},
\]

\[
\bar{T}_{jk}^{E}(\boldsymbol{x}_{m},\boldsymbol{\xi}_{l})=\pm\dfrac{1}{2}\delta_{jk}\delta_{ml}+\int_{\Delta S_{l}}T_{jk}^{E}(\boldsymbol{x}_{m},\boldsymbol{\xi})dS_{\boldsymbol{\xi}}
\]

\[
\bar{G}_{jk}^{I}(\boldsymbol{x}_{m},\boldsymbol{\xi}_{l})=\int_{\Delta S_{l}}G_{jk}^{I}(\boldsymbol{x}_{m},\boldsymbol{\xi})dS_{\boldsymbol{\xi}},
\]

\[
\bar{T}_{jk}^{I}(\boldsymbol{x}_{m},\boldsymbol{\xi}_{l})=\pm\dfrac{1}{2}\delta_{jk}\delta_{ml}+\int_{\Delta S_{l}}T_{jk}^{I}(\boldsymbol{x}_{m},\boldsymbol{\xi})dS_{\boldsymbol{\xi}}
\]

\[
\mathcal{L}\boldsymbol{u}+\boldsymbol{f}=0
\]

\[
\mathcal{L}\boldsymbol{u}=\omega^{2}\boldsymbol{u}+c_{p}^{2}\nabla(\nabla\cdot\boldsymbol{u})-i\omega\dfrac{\mu}{\rho}\nabla^{2}\boldsymbol{u}
\]
\begin{align*}
\int v_{j}(\mathcal{L}u_{i})_{j}-u_{j}(\mathcal{L}v_{i})_{j}dV & =c_{p}^{2}\int v_{j}u_{i,ij}-u_{j}v_{i,ij}dV-i\omega\dfrac{\mu}{\rho}\int v_{i}u_{i,jj}-u_{i}v_{i,jj}dV\\
 & =c_{p}^{2}\int(\nabla\cdot\boldsymbol{u})\boldsymbol{n}\cdot\boldsymbol{v}-(\nabla\cdot\boldsymbol{v})\boldsymbol{n}\cdot\boldsymbol{u}dS-i\omega\dfrac{\mu}{\rho}\int\boldsymbol{v}\nabla\boldsymbol{u}-\boldsymbol{u}\nabla\boldsymbol{v}dS,\\
 & \ (Green's\ Idenities)
\end{align*}

\begin{align*}
\alpha u(\boldsymbol{x}) & =\int_{\Omega}f(\boldsymbol{\xi})G(\boldsymbol{x},\boldsymbol{\xi})d\boldsymbol{\xi}+\\
 & \int_{S}\boldsymbol{u}(\boldsymbol{\xi})[c_{p}^{2}(\nabla\cdot G(\boldsymbol{x},\boldsymbol{\xi}))\boldsymbol{n}-i\omega\dfrac{\mu}{\rho}\nabla G(\boldsymbol{x},\boldsymbol{\xi})]-G(\boldsymbol{x},\boldsymbol{\xi})[c_{p}^{2}(\nabla\cdot\boldsymbol{u}(\boldsymbol{\xi}))\boldsymbol{n}-i\omega\dfrac{\mu}{\rho}\nabla\boldsymbol{u}(\boldsymbol{\xi})]dS_{\xi}
\end{align*}


\section*{APPENDIX}

\subsection*{Vector Denotation and Identities}

consider scalar $u$ and vector $\boldsymbol{V}$: 

\[
grad\ u=\nabla u=u_{,i}=\dfrac{\partial u}{\partial x_{i}}
\]

\[
div\ u=\nabla\cdot u=u_{i,i}
\]

\[
\nabla\cdot(\nabla u)=\nabla^{2}u=u_{,ii}
\]

\[
\nabla\cdot(\nabla\boldsymbol{V})=\nabla^{2}\boldsymbol{V}=\nabla^{2}v_{i}=v_{i,jj}
\]

\[
\nabla\times\boldsymbol{V}=\begin{vmatrix}\boldsymbol{i} & \boldsymbol{j} & \boldsymbol{k}\\
\dfrac{\partial}{\partial x} & \dfrac{\partial}{\partial y} & \dfrac{\partial}{\partial z}\\
u & v & w
\end{vmatrix}
\]

\[
\nabla\times\nabla u=0,\ \text{curl of a gradient is zero}
\]

\[
\nabla\cdot(\nabla\times\boldsymbol{V})=0,\ \text{divergence of a curl is zero }
\]

\[
\nabla^{2}\boldsymbol{V}=\nabla(\nabla\cdot\boldsymbol{V})-\nabla\times(\nabla\times\boldsymbol{V}),\!\ \ \nabla\times\nabla\times(\nabla\times\boldsymbol{V})=-\nabla\times(\nabla^{2}\boldsymbol{V}).
\]

\[
\text{divergency theorem: }\int_{\Omega}\nabla\cdot\boldsymbol{V}dA=\int_{\partial\Omega}\boldsymbol{V}\cdot\boldsymbol{n}ds
\]

\end{CJK}

\end{document}
